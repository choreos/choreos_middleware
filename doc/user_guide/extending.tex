
\chapter{Extending \ee}

\section{Introduction}

Current PaaS solutions available on market are well known for their low flexibility. Some of them work only on a specific cloud environment, others only with a few development frameworks. \ee\ tries to overcome this issue by providing an extensible architecture. Although the out-of-box version of \ee\ is quite limited, with some programming is possible to extend it to provide support to new \emph{i)} cloud providers, \emph{ii)} package types, \emph{iii)} service types, and \emph{iv)} node selection policies. By ``extending'' we mean no current \ee\ code need to be changed, and that each new extension can be implemented by the means of a well-defined process, which are now described in this section.

\section{Supporting new cloud providers}

In \ee, cloud providers are just a source of virtual machines provisioning. Any technology able to create new virtual machines may be used as ``cloud provider''. 

To implement a new cloud provider, it is necessary to implement the \textsf{CloudProvider} interface (Listing~\ref{lst:cloud_provider}). Current implementations are \textsf{AWSCloudProvider} (that uses EC2 service), \textsf{OpenStackKeystoneCloudProvider}, and \textsf{FixedCloudProvider} (that always points to the same user-defined VMs). An example of new cloud provider implementation could be the \textsf{VirtualBoxCloudProvider}, that would use VirtualBox on the developer machine to create new VMs (this is an example more suited to test environments).

\lstset{
language=Java,
numbers=left
}

{\footnotesize
\begin{lstlisting}[caption=\textsf{CloudProvider} interface, label=lst:cloud_provider]
public interface CloudProvider {

    public String getCloudProviderName();

    public CloudNode createNode(NodeSpec nodeSpec) 
    	throws NodeNotCreatedException;

    public CloudNode getNode(String nodeId) 
    	throws NodeNotFoundException;

    public List<CloudNode> getNodes();

    public void destroyNode(String id) 
    	throws NodeNotDestroyed, NodeNotFoundException;

    public CloudNode createOrUseExistingNode(NodeSpec nodeSpec) 
    	throws NodeNotCreatedException;
}
\end{lstlisting}
}

\emph{Important note}: in the current implementation, \ee\ is tailored to work with nodes running \emph{Ubuntu 12.04}. Therefore, \textsf{CloudProvider} implementors should provide Ubuntu 12.04 nodes. 

The next step is to edit the \texttt{cloud\_providers.properties} file, located on DeploymentManager resources folder. You must add a line in the format \verb!NAME=full.qualified.class.name!, where the key is just an alias that you can freely define (since it does not conflict with other existing keys on the same file), and the value is the full qualified name of the \textsf{CloudProvider} implementing class. It is also necessary to recompile the DeploymentManager project in such way it can access the implementing class.

Finally, to use your new cloud provider, it is enough to configure the \texttt{deployment.properties}: add the line \verb!CLOUD_PROVIDER=NAME!, where \verb!NAME! is the \verb!NAME! defined in the \texttt{cloud\_providers.properties} file (be sure there is no other line defining the \verb!CLOUD_PROVIDER! property).

\section{Supporting new package types}

Services may be delivered in different package types, such as JARs, WARs, etc.
Each package type has its own specific deployment procedures, as well its specific process to start the service.
When using different technologies, such as Python, to write new services, you will need to define a new package type,
as well the deployment procedure associated with it. Such procedure is specified in Chef recipe. 

So, the first step is to create a new Chef recipe similar to the ``jar'' and ``war'' recipes already provided by \ee.
This recipes are actually templates that \ee\ will use to create specific recipes to each service to be deployed.
You may use the the constants \$PACKAGE\_URL and \$NAME within your recipe, they will be injected by \ee to each specific recipe.
After writing the new recipe, you must associate this recipe to the new package type by editing the \texttt{cookbooks.properties} file.

% TODO nativeUri! 

%criar receita, tipo a do jar
%variáveis que serão susbtituídas:
%\$NAME na receita
%e \$PACKAGE\_URL e \$NAME nos atributos
%(no futuro devemos passar todos os atributos do service spec)
%
%editar cookbooks.properties
%
%nativeUri -> vai ter q criar mais uma classe pra ser extendida
%e é preciso um arquivo mapeando o packageType pra classe que define as uris (uricontexts)
%defaultPort -> criar arquivo de portas padrões (ports)
%Tanto uricontexts quanto ports são arquivos que vão ser lidos pelas classes em EnactmentEngineAPI
%Mas se os arquivos ficarem nesse projeto, aí eles não serão lidos corretamente quando 
%essas classes forem chamadas no DeploymentManager.
%Que fazer? =T
%
%usando novo package type no service spec
%
%dica: pode ser interessante criar uma nova imagem
%com o middleware do pacote correspondente já instalado


\section{Supporting new service types}

Although web services came to tackle the interoperability issue, today we have a couple of technologies implementing the concept of services.
The main technologies in this context are SOAP and RESTful services, but other technologies could be used to implement services, such as JMS.

In the \ee\ context, the service type affects only how the \texttt{setInvocationAddress} is invoked.
Therefore, to support a new service type, you have only to write a new \textsf{ContextSender} (Listing~\ref{lst:context_sender}) implementation.

{\footnotesize
\begin{lstlisting}[caption=\textsf{ContextSender} interface, label=lst:context_sender]
public interface ContextSender {

    /**
     * Calls setInvokationAddress operation on service in the serviceEndpoint.
     * So, the service in endpoint will know that its partner named partnerName
     * with partnerRole is realized by instances in partnerEndpoints.
     * 
     */
    public void sendContext(String serviceEndpoint, String partnerRole, String partnerName,
            List<String> partnerEndpoints) throws ContextNotSentException;
}
\end{lstlisting}
}


\section{Supporting new node selection policies}

A node selection policy defines the \emph{mapping} of services to cloud nodes.
Since cloud nodes are dynamically created, node selection policies must be flexible, and not rely on hard-coded IPs.
A node selection policy may define nodes to be used based on services non-functional properties.
To create a new node selector, you must create a new \textsf{NodeSelector} (Listing~\ref{lst:node_selector}) implementation.
Pay attention that such implementation must be thread-safe, since multiple threads will invoke concurrently the method \texttt{select}.

{\footnotesize
\begin{lstlisting}[caption=\textsf{NodeSelector} interface, label=lst:node_selector]
/**
 * Selects a node to apply a given configuration
 * 
 * The selection can consider functional requirements, which is provided by
 * spec.resourceImpact. Implementing classes must use the NodePoolManager to
 * retrieve nodes AND/OR create new nodes. NodeSelectors are always accessed as
 * singletons. Implementing classes must consider concurrent access to the
 * select method.
 * 
 * @author leonardo
 * 
 */
public interface NodeSelector extends Selector<CloudNode, DeployableServiceSpec> {

}

/**
 * Selects objects from a given source according to the requirements. If
 * necessary, creates new objects using a given factory.
 * 
 * @author leonardo
 * 
 * @param <T>
 *            the class of the selected resource
 * @param <R>
 *            the class of the requirements
 */
public interface Selector<T, R> {

    public List<T> select(R requirements, int objectsQuantity) throws NotSelectedException;

}
\end{lstlisting}
}

After writing the new node selector, you must associate this selector to a label by editing the \texttt{node\_selector.properties} file.
To use the new selector, finally, you must attribute the defined label to the NODE\_SELECTOR property on the \texttt{deployment.properties} file.







