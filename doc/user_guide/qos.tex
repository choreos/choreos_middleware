\chapter{Elasticity and QoS management}
\chaptermark{Elasticity QoS Management}

As mentioned in section \ref{chap:rest-api}, the \ee is able to modify a
choreography that is currently running.
%
For instance, one may decide to switch from using a service offered by one
provider to a compatible service by a different provider, or to
increase/decrease the number of deployed replicas of a given service in order
to adapt to fluctuations in usage load.
%
To do this, the user simply uses the API again to submit an updated version of the
choreography specification to the \ee and requests the reenactment of the
choreography.
%
The \ee, in turn, detects the modifications made to the choreography and
performs the requested modifications, by deploying new versions of services,
removing service replicas etc.

This capability, together with the flexibility offered by the CHOReOS
monitoring subsystem, presents the user with the framework necessary to adjust
the run-time environment of the choreography according to QoS parameters and
constraints, such as response time or cost.
%
In order to accomplish this, the user needs to create a separate daemon that
acts both as a monitoring client and as a client for the \ee.
%
As a client for the monitoring system, this daemon uploads rules to the
Glimpse Monitoring CEP and awaits for notifications from it whenever such
rules are triggered;
%
as a client for the \ee, it requests modifications to running choreographies
by sumitting updated Choreography Deployers when notifications are received
from the monitoring subsystem.

An example of such a daemon is available in the \ee source code repository,
in the ``reconfiguration'' directory. We show below some code snippets from
this example daemon and explain its general mechanism.

During startup, the daemon loads predefined rules from a static file and
submits them to the Glimpse monitor:

\begin{verbatim}
public static void main(String[] args) {
    String rules = Manager.ReadTextFromFile(
        this.getClass().getClassLoader().getResource("rules/SLAViolations.xml").getFile());
    new EnactmentEngineGlimpseConsumer([... properties ...], rules);
}
\end{verbatim}

Whenever a rule is triggered, the daemon is notified and runs the code
from a class whose name is contained in the notification event:

\begin{verbatim}
public void messageReceived(Message arg0) { 
    ObjectMessage responseFromMonitoring = (ObjectMessage) arg0;
    response = (ComplexEventResponse) responseFromMonitoring.getObject();
    event = new HandlingEvent(response.getResponseValue(), response.getRuleName());
    Class<ComplexEventHandler> theClass;
    theClass = (Class<ComplexEventHandler>) Class.forName(
        "org.ow2.choreos.chors.reconfiguration.events." + event.getEventData());
    handler = theClass.newInstance();
    handler.handleEvent(event);
}
\end{verbatim}

The submitted example rules define that whenever more than 5\% of requests
during the last 2 minutes have a response time above 120 miliseconds, new replicas of
the service should be created (the class \textsf{AddReplica} should be run):

\begin{verbatim}
    when
        $ev : ResponseTimeEvent() over window:time(2m);
        
        Number( $eventSum : doubleValue ) from accumulate(
            $event : ResponseTimeEvent($ev.service == service, $ev.chor == chor, $ev.ip == ip)
                over window:time(2m), count($event)
        );
            
        Number( intValue > $eventSum*0.05 ) from accumulate(
            $sEvent : ResponseTimeEvent(
                value > 120, $ev.service == service, $ev.chor == chor, $ev.ip == ip)
                over window:time(2m), count($sEvent)
        );

    then
        ResponseDispatcher.NotifyMeValue("AddReplica",
            "eeConsumer", (String) $ev.ip, (String) $ev.service);
    end
\end{verbatim}

Finally, the \textsf{AddReplica} class then interacts with the \ee to update the
number of replicas of the service:

\begin{verbatim}
public void handleEvent(HandlingEvent event) {
    List<DeployableService> services = registryHelper.getServicesHostedOn(event.getNode());
    List<DeployableServiceSpec> serviceSpecs = registryHelper.getServiceSpecsForServices(services);

    Choreography c = registryHelper.getChor(event.getNode());
    ChoreographySpec cSpec = c.getChoreographySpec();

    for (DeployableServiceSpec spec : serviceSpecs) {
        for (DeployableServiceSpec s : cSpec.getDeployableServiceSpecs()) {
            if (s.getName().equals(spec.getName())) {
                s.setNumberOfInstances(s.getNumberOfInstances() + 1);
                break;
            }
        }
    }

    registryHelper.getChorClient().updateChoreography([... id ...], cSpec);
    registryHelper.getChorClient().enactChoreography([... id ...]);
}
\end{verbatim}
